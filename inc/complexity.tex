
\subsection{計算量}
\frame{
  \frametitle{計算量}
  \begin{block}{ランダウの記号}
    \[f(x) \in O(g(x)) \Leftrightarrow \lim_{x \rightarrow \infty}{\frac{f(x)}{g(x)} < \infty} \]
    \[f(x) \in \Omega(g(x)) \Leftrightarrow \lim_{x \rightarrow \infty}{\frac{f(x)}{g(x)} > 0} \]
    \[f(x) \in \Theta(g(x)) \Leftrightarrow 0 < \lim_{x \rightarrow \infty}{\frac{f(x)}{g(x)} < \infty} \]
    \[f(x) \in o(g(x)) \Leftrightarrow \lim_{x \rightarrow \infty}{\frac{f(x)}{g(x)} = 0} \]
  \end{block}
}

\frame{
  \center{
    \includegraphics<1>[width=8cm]{image/complexity01.pdf}
    \includegraphics<2>[width=8cm]{image/complexity02.pdf}
  }
}

\frame{
  \frametitle{計算量}
  \begin{alertblock}{ランダウの記号}
    \center{
      $f(x) \in O(g(x))$ を $f(x) = O(g(x))$ と書く事がある!!\\
    }
  \end{alertblock}
  \begin{itemize}
  \item いろいろと便利
  \item しかしこれはどうなんだ
  \end{itemize}
}
