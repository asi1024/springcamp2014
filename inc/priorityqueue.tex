
\subsection{プライオリティキュー}
\frame{
  \frametitle{プライオリティキュー}
  \begin{block}{Priority Queue}
    全順序集合$P(C,\leq)$について,$C$上の要素からなる\\
    プライオリティキューは以下の操作が実行できる.
    \itemize{
    \item push : $C$の要素$x$を追加する
    \item pop : ヒープの最大値をもつ要素を削除する
    \item top : ヒープの最大値を求める\\
    }
  \end{block}
  \pause
  \begin{alertblock}{}
    様々なアルゴリズムに応用できる!!
  \end{alertblock}
}

\frame{
  \frametitle{ナイーブな実装}
  \begin{block}{リストによる実装}
    \only<2->{前もって最大値を計算しておけば?}
    \itemize{
    \item push : O(1)で$C$の要素$x$を追加する
    \item pop : O(n)でヒープの最大値をもつ要素を削除する
    \only<1>{\item top : O(n)でヒープの最大値を求める\\}
    \only<2->{\item top : O(1)でヒープの最大値を求める\\}
    }
  \end{block}
  \only<3->{こんなもん使えるか!!} \\
  \only<4>{ここで,人類は「ヒープ」を発明したのだ} \\
}
