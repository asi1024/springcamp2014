
\subsection{最短経路問題}
\frame{
  \frametitle{最短経路問題}
  \begin{block}{最短経路問題}
    \begin{itemize}
    \item インスタンス \\
      有向グラフ$G$,重み関数$c:E(G)\rightarrow\mathbb{R}$,2点$s,t\in V$ \\ 
    \item タスク \\
      最短$s-t-$パス$P$ or $s$から$t$が到達不可能であることの決定
    \end{itemize}
  \end{block}
}

%-----------------------------------------------------------

\frame{
  \frametitle{Dijkstraのアルゴリズム}
  単一始点全点間最短距離を求めるアルゴリズム
  \begin{block}{Dijkstraのアルゴリズム}
    \begin{itemize}
    \item 入力 \\
      重み付き有向グラフ$G$\\
      重み関数$c:E(G)\rightarrow\mathbb{R_+}$\\
      始点$s\in V(G)$ \\ 
    \item 出力 \\ $s$から全ての$t\in V(G)$へのパスとその距離
    \end{itemize}
  \end{block}
}

\frame{
  \begin{block}{Dijkstraのアルゴリズム}
    1.  $d(s)\leftarrow 0, \forall v\in V(G)-\{s\}~d(v)\leftarrow\infty, R\leftarrow\emptyset$ で初期化\\
    2.  $d(v)=\displaystyle\min_{w\in V(G)-R}$となる$v\in V(G)-R$を1つ求める\\
    3.  $R\leftarrow R\cup \{v\}$ とする\\
    4.  {\bf For} $(v,w)\in E$ を満たす全ての$w\in V(G)-R$ {\bf do}\\
    ~~~~~ {\bf If} $d(w)>d(v)+c((v,w))$ {\bf then} \\
    ~~~~~~~~ $d(w)\leftarrow d(v)+c((v,w)),~p(w)\leftarrow v$ とする \\
    5.  {\bf If} $R\neq V(G)$ {\bf then} 2に行く
  \end{block}
  $R$は最短路が確定した頂点\\
  2〜5のループは高々$|V(G)|$回
}

\frame{
  \begin{exampleblock}{例}
    \center{
      \includegraphics<1>[width=7cm]{image/dijkstra01.pdf}
      \includegraphics<2>[width=7cm]{image/dijkstra02.pdf}
      \includegraphics<3>[width=7cm]{image/dijkstra03.pdf}
      \includegraphics<4>[width=7cm]{image/dijkstra04.pdf}
      \includegraphics<5>[width=7cm]{image/dijkstra05.pdf}
      \includegraphics<6>[width=7cm]{image/dijkstra06.pdf}
      \includegraphics<7>[width=7cm]{image/dijkstra07.pdf}
      \includegraphics<8>[width=7cm]{image/dijkstra08.pdf}
      \includegraphics<9>[width=7cm]{image/dijkstra09.pdf}
      \includegraphics<10>[width=7cm]{image/dijkstra10.pdf}
      \includegraphics<11>[width=7cm]{image/dijkstra11.pdf}
      \includegraphics<12>[width=7cm]{image/dijkstra12.pdf}
      \includegraphics<13>[width=7cm]{image/dijkstra13.pdf}
    }
  \end{exampleblock}
}

\frame{
  \begin{block}{Dijkstraのアルゴリズム}
    1.  $d(s)\leftarrow 0, \forall v\in V(G)-\{s\}~d(v)\leftarrow\infty, R\leftarrow\emptyset$ で初期化\\
    2.  $d(v)=\displaystyle\min_{w\in V(G)-R}$となる$v\in V(G)-R$を1つ求める\\
                  
    \only<1,2>{\alert{ナイーブな実装では$O(V)$}}
    \only<3,4>{\alert{top,popにかかる時間は$O(\log V)$}}\\
    3.  $R\leftarrow R\cup \{v\}$ とする\\
    4.  {\bf For} $(v,w)\in E$ を満たす全ての$w\in V(G)-R$ {\bf do}\\
    ~~~~~ {\bf If} $d(w)>d(v)+c((v,w))$ {\bf then} \\
    ~~~~~~~~ $d(w)\leftarrow d(v)+c((v,w)),~p(w)\leftarrow v$ とする\\
                    
    \only<3>{\alert{pushで$O(\log V)$}}
    \only<4>{\alert{push,decreasingKeyで$O(1)$}}\\
    5.  {\bf If} $R\neq V(G)$ {\bf then} 2に行く
  \end{block}
  2〜5のループは高々$|V(G)|$回\\
  \only<1,2>{計算量は$O(V^2)$}
  \only<3>{計算量は$O((E+V)\log V)$}
  \only<4>{計算量は$O(E+V\log V)$}
    
  \only<2>{\alert{ここでヒープを使う!!!}}
  \only<3>{\alert{二分ヒープを使った場合}}
  \only<4>{\alert{フィボナッチヒープを使った場合}}
}

\frame{
  \frametitle{Dijkstraのアルゴリズムの利用例}
  \begin{block}{Dijkstraのアルゴリズムの利用例}
    カーナビの経路探索や鉄道の経路案内において利用されている.
    ゲームのAIとか作るのにも使うかも
  \end{block}
}
